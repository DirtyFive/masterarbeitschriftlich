\documentclass{llncs}
\usepackage[utf8]{inputenc}				%% Für Umlaute in UTF-8-Kodierung
\usepackage{amssymb}					%% Für alle denkbaren mathematischen Symbole
\usepackage[german,english]{babel}		%% Für deutsche (und englische) Trennungsregeln und Bezeichnungen
\usepackage{hyperref}					%% Für Hyperlinks im PDF zum Anklicken
\usepackage{apacite}					%% Für Zitationsstil gem. APA

\begin{document}
\selectlanguage{german}
\title{Hamiltonian Mechanics unter besonderer Ber\"ucksichtigung der h\"ohreren Lehranstalten}
\author{Susanne Klickerklacker}
\institute{Seminar KI: gestern, heute, morgen\\ Angewandte Informatik, Universität Bamberg}
\maketitle								%% Titel setzen

\begin{abstract}
The abstract should summarize the contents of the paper.
\keywords{Einordnung gemäß ACM + 1--2 weitere Schlüsselwörter} %% siehe http://delivery.acm.org/10.1145/2380000/2371137/ACMCCSTaxonomy.html
\end{abstract}
%
\section{Einleitung}
%
Es beginnt mit der einfachen Differentialgleichung
\begin{eqnarray}
  \dot{x}&=&JH' (t,x)\label{gleichung} \\
  x(0)   &=& x(T). \label{nebenbedingung}
\end{eqnarray}

Das Lösen dieser Gleichung ist ein wichtiger Baustein zum Verständnis der Mechanik \cite{mechanics}.
Zunächst eine Definition.

\begin{definition}
Eine Zahl $x \in \mathbb{R}$ heißt \emph{groß}, falls $x>2$ gilt.
\label{def:gross}
\end{definition}

Dieser Artikel ist eine Beschreibung eines algorithmischen Verfahrens zur Lösung von Gleichung~\ref{gleichung} unter Nebenbedingung~\ref{nebenbedingung}. 
Ein Ablaufdiagramm des Verfahrens ist in Abb.~\ref{fig:diagramm}) dargestellt. 
Der hier vorgestellte Ansatz ist nicht anwendbar für Zahlen die gross sind im Sinne von Definition~\ref{def:gross}, damit komplementiert er die von \citeA{bob} vorgestellten Techniken, die nur für im selbem Sinne grossen Zahlen anwendbar sind.


\section{Hinweise}
\begin{enumerate}
	\item Verwendung der Stilvorgaben (diese Vorlage) ist \emph{obligatorisch} -- so üben Sie, für eine wissenschaftliche Konferenz Papiere vorzubereiten. Die hier verwendete Vorlage basiert auf der letzten KI-Tagung. 
	
	\item CitH-Studierende im ersten Semester dürfen mit Word o.ä. arbeiten, nutzen Sie dazu zwingenderweise\ldots
	\begin{enumerate}
		\item die Formatvorlage der KI ist die LLNCS-Vorlage\footnote{\scriptsize siehe \url{http://www.springer.com/de/it-informatik/lncs/conference-proceedings-guidelines}}
		\item Zitationsstil gemäß der APA, siehe \url{http://www.apastyle.org/}
		\item eine Software zum Generieren konsistenter Literaturreferenzen (z.B. EndNote, RefWorks, Citavi) die ein Plug-In für die von Ihnen verwendete Textverarbeitung bereitstellt
	\end{enumerate}
	
	\item Ihre Ausarbeitung soll 15 Seiten \emph{nicht} überschreiten!
	
	\item Auch Latex und Bibtex garantieren keine Konsistenz und Vollständigkeit Ihren Angaben: beachten Sie beim Aufruf von Latex und Bibtex auftretende Warnungen.
	
	\item Beachten Sie die Möglichkeiten, die durch die Formatvorlagen definiert werden (siehe Dokumentation LLNCS bzw. apacite\footnote{\scriptsize siehe \url{ftp://ftp.dante.de/tex-archive/biblio/bibtex/contrib/apacite/apacite.pdf}}
\end{enumerate}

\begin{figure}
\centering
\framebox{\huge A} $\to$ \framebox{\huge B}
\caption{\label{fig:diagramm}Ablaufdiagramm der betrachteten Architektur}
\end{figure}

\bibliographystyle{apacite}
\bibliography{mybib.bib}
\end{document}
\end

Alles was hinter \end steht wird von Latex ignoriert...